\documentclass[11pt,twoside,reqno]{amsart}
\pdfoutput=1
\usepackage{amssymb,amsthm,amsthm,amsfonts,latexsym}
\usepackage{amsmath}
\usepackage{mathrsfs}
\usepackage{stmaryrd}
\usepackage{accents}
\usepackage[latin1]{inputenc}
\usepackage{color}
\usepackage{xspace}
\usepackage{tikz-cd}
\usepackage{float}
\usepackage[]{hyperref}
\usepackage[]{cleveref}

\usepackage[colorinlistoftodos,prependcaption,textsize=tiny,spanish,textwidth=2cm]{todonotes}\setlength{\marginparwidth}{0.8in}

\allowdisplaybreaks

\setlength{\textwidth}{16cm}
\setlength{\topmargin}{0cm}
\setlength{\oddsidemargin}{.5cm}
\setlength{\evensidemargin}{.5cm}
\setlength{\textheight}{21.5cm}

\theoremstyle{plain}
\newtheorem{theorem}{Theorem}[section]
\newtheorem*{theorem*}{Theorem}
\newtheorem{lemma}[theorem]{Lemma}
\newtheorem{corollary}[theorem]{Corollary}
\newtheorem{proposition}[theorem]{Proposition}
\newtheorem{conjecture}[theorem]{Conjecture}
%\newtheorem*{atheorem}{Theorem \ref{prank3}}
%\newtheorem*{bproposition}{Proposition \ref{propContracibilitySufConditions}}
%\newtheorem*{cproposition}{Proposition \ref{propositionApGContractibilityLowSteps}}
\theoremstyle{definition}
\newtheorem{example}[theorem]{Example}
\newtheorem{definition}[theorem]{Definition}

\theoremstyle{remark}
\newtheorem{remark}[theorem]{Remark}

\crefname{theorem}{Theorem}{Theorems}
\crefname{lemma}{Lemma}{Lemmas}
\crefname{proposition}{Proposition}{Propositions}
\crefname{corollary}{Corollary}{Corollaries}
\crefname{definition}{Definition}{Definitions}
\crefname{remark}{Remark}{Remarks}
\crefname{conjecture}{Conjecture}{Conjectures}



\def\co{\colon}
\def\wt{\widetilde}
\newcommand{\tq}{\mathrel{{\ensuremath{\: : \: }}}}
\def\op{\mathrm{op}}
\newcommand{\rk}[1]{\mathrm{rk}\left( #1 \right)}
\def\Im{\mathrm{Im}}

%\let\emptyset\varnothing
\def\Aut{\mathrm{Aut}}
\def\A{\mathcal{A}}
\def\B{{\mathcal B}}
\def\F{\mathcal{F}}
\def\K{\mathcal{K}}
\def\S{\mathcal{S}}
\def\Z{\mathbb{Z}}
\def\SLV{\mathcal{SLV}}
\def\PSL{\mathrm{PSL}}
\def\Sz{\mathrm{Sz}}

\pagestyle{myheadings} \markboth{{\sc  k.i. piterman, i. sadofschi costa, a. viruel}}{{\sc Quillen's conjecture for groups of p-rank 3.}}


\begin{document}

\title[Quillen's conjecture for groups of p-rank 3]{Quillen's conjecture for groups of $p$-rank 3}

\author[K.I. Piterman]{Kevin Iv\'an Piterman$^{*}$}
\author[I. Sadofschi Costa]{Iv\'an Sadofschi Costa$^{**}$}
\author[A. Viruel]{Antonio Viruel$^{***}$}

\thanks{This work was partially done at the University of M\'alaga, during a research stay of the first two authors, supported by project MTM2016-78647-P.\\
$*$ Supported by a CONICET doctoral fellowship and grants CONICET PIP 11220170100357CO and UBACyT 20020160100081BA.\\
$**$ Supported by a CONICET doctoral fellowship and grants ANPCyT PICT-2017-2806, CONICET PIP 11220170100357CO and UBACyT 20020160100081BA \\
$***$ Supported by...}
\address{Universidad de Buenos Aires. Facultad de Ciencias Exactas y Naturales. Departamento de Matem\'atica. Buenos Aires, Argentina.}

\address{CONICET-Universidad de Buenos Aires. Instituto de Investigaciones Matem\'aticas Luis A. Santal\'o (IMAS). Buenos Aires, Argentina.}

\email{kpiterman@dm.uba.ar}
\email{isadofschi@dm.uba.ar}
\email{viruel@uma.es}

\begin{abstract}
Let $G$ be a finite group and $\A_p(G)$ be the poset of nontrivial elementary abelian $p$-subgroups of $G$. Quillen conjectured that $O_p(G)$ is nontrivial if $\A_p(G)$ is contractible. We prove Quillen's conjecture for groups of $p$-rank $3$.
\end{abstract}


\subjclass[2010]{57S17, %Finite transformation groups
		20D05, %Classification of simple and nonsolvable groups
		57M20, %Two-dimensional complexes
		55M20, %Fixed points and coincidences
		55M35, %Finite groups of transformations (including Smith theory)
		57M60 %Group actions in low dimensions
		}


\keywords{Quillen's conjecture, poset, p-subgroups}

\maketitle

\section{Introduction}

The poset $\S_p(G)$ of nontrivial $p$-subgroups of $G$ was introduced by K.S. Brown in \cite{Brown}, where he proved that the Euler characteristic $\chi(\K(\S_p(G)))$ of its order complex is $1$ modulo the greatest power of $p$ dividing the order of $G$.
Some years later, Quillen \cite{Q} studied some homotopy properties of $\K(\S_p(G))$.
In that article, Quillen considered the subposet $\A_p(G)$ of nontrivial elementary abelian $p$-subgroups and proved that it is homotopy equivalent to $\S_p(G)$ \cite[Proposition 2.1]{Q}.

Quillen also proved that if $O_p(G)$, the greatest normal $p$-subgroup of $G$,  is nontrivial then $\A_p(G)$ is contractible \cite[Proposition 2.4]{Q} and conjectured that the converse should hold.
In this paper we consider the following stronger version of Quillen's conjecture, stated by Aschbacher and Smith \cite{AschbacherSmith}.

\begin{conjecture}[Quillen's conjecture]\label{QCAschbacherSmith}
If $O_p(G)=1$ then $\wt{H}_*(\A_p(G))\neq 0$.
\end{conjecture}

Quillen proved some cases of this conjecture. For example, he proved it for solvable groups \cite[Theorem 12.1]{Q}.
In \cite{AschbacherSmith}, M. Aschbacher and S.D. Smith made a huge progress on the study of this conjecture.
By using the classification of finite simple groups, they proved that Quillen's conjecture holds if $p>5$ and $G$ does not contain certain unitary components.
Previously, Aschbacher and Kleidman \cite{AK} had proved Quillen's conjecture for almost simple groups (i.e. finite groups $G$ such that $L\leq G\leq \Aut(L)$ for some simple group $L$).

In this paper we prove Quillen's conjecture \ref{QCAschbacherSmith} for groups of $p$-rank $3$.
Recall that the \textit{$p$-rank} of $G$ is the maximum possible rank of an elementary abelian $p$-subgroup of $G$.
The $p$-rank $2$ case was considered by Quillen \cite[Proposition 2.10]{Q} and follows from the fact that an action of a finite group on a tree has a fixed point.

C. Casacuberta and W. Dicks conjectured that every finite group acting on a contractible $2$-complex has a fixed point \cite{CD}.
This conjecture was studied by Aschbacher and Segev in \cite{AschbacherSegev}.
Posteriorly, Oliver and Segev classified groups that can act without fixed points on an acyclic $2$-complex. The results of their work \cite{OS} are the basis of our proof of the $p$-rank $3$ case of Quillen's conjecture. Our main result, \cref{prank3} can also be seen as a special case of the Casacuberta-Dicks conjecture.

Note that  \cref{prank3} was not known, for the results of \cite{AschbacherSmith} do not apply for $p<5$.\todo{I: ahora tenemos ejemplos para afirmar esto con m�s confianza no?}

\section{The results of Oliver and Segev}

In this section we review the results of \cite{OS} needed in the proof of \cref{prank3}.
If $X$ is a poset, $\K(X)$ denotes the order complex of $X$ (i.e. the simplicial complex whose simplices are the finite nonempty totally ordered subsets of $X$).
By a $G$-complex we mean a $G$-CW complex. Note that the order complex of a $G$-poset is always a $G$-complex.

\begin{definition}[{\cite{OS}}]
 A $G$-complex $X$ is \textit{essential} if there is no normal subgroup $1\neq N\triangleleft G$ such that for each $H\subseteq G$, the inclusion $X^{HN}\to X^H$ induces an isomorphism on integral homology.
\end{definition}

The main results of \cite{OS} are the following two theorems.

\begin{theorem}[{\cite[Theorem A]{OS}}]\label{teoA}
 For any finite group $G$, there is an essential fixed point free $2$-dimensional (finite) $\Z$-acyclic $G$-complex if and only if $G$ is isomorphic to one of the simple groups $\PSL_2(2^k)$ for $k\geq 2$, $\PSL_2(q)$ for $q\equiv \pm 3 \pmod 8$ and $q\geq 5$, or $\Sz(2^k)$ for odd $k\geq 3$. Furthermore, the isotropy subgroups of any such $G$-complex are all solvable.
\end{theorem}

\begin{theorem}[{\cite[Theorem B]{OS}}]\label{teoB}
 Let $G$ be any finite group, and let $X$ be any $2$-dimensional $\Z$-acyclic $G$-complex. Let $N$ be the subgroup generated by all normal subgroups $N'\triangleleft G$ such that $X^{N'}\neq \emptyset$. Then $X^N$ is $\Z$-acyclic; $X$ is essential if and only if $N=1$; and the action of $G/N$ on $X^N$ is essential.
\end{theorem}

The set of subgroups of $G$ will be denoted by $\S(G)$.

\begin{definition}[{\cite{OS}}]
 By a \textit{family} of subgroups of $G$ we mean any subset $\F\subseteq\S(G)$ which is closed under conjugation.
 A nonempty family is said to be \textit{separating} if it has the following three properties: (a) $G\notin \F$; (b) if $H'\subseteq H$ and $H\in \F$ then $H'\in \F$; (c) for any $H\triangleleft K\subseteq G$ with $K/H$ solvable, $K\in \F$ if $H\in \F$.

 For any family $\F$ of subgroups of $G$, a \textit{$(G,\F)$-complex} will mean a $G$-complex all of whose isotropy subgroups lie in $\F$. A $(G,\F)$-complex is \textit{$H$-universal} if the fixed point set of each $H\in \F$ is acyclic.
\end{definition}

\begin{lemma}[{\cite[Lemma 1.2]{OS}}]\label{lemma1.2}
 Let $X$ be any $2$-dimensional acyclic $G$-complex without fixed points. Let $\F$ be the set of subgroups $H\subseteq G$ such that $X^H\neq \emptyset$. Then $\F$ is a separating family of subgroups of $G$, and $X$ is an $H$-universal $(G,\F)$-complex.
\end{lemma}

If $G$ is not solvable, the separating family of solvable subgroups of $G$ is denoted by $\SLV$.

\begin{proposition}[{\cite[Proposition 6.4]{OS}}]\label{proposition6.4}
 Assume that $L$ is one of the simple groups $\PSL_2(q)$ or $\Sz(q)$, where $q=p^k$ and $p$ is prime ($p=2$ in the second case). Let $G\subseteq\Aut(L)$ be any subgroup containing $L$, and let $\F$ be a separating family for $G$. Then there is a $2$-dimensional $\Z$-acyclic $(G,\F)$-complex if and only if $G=L$, $\F=\SLV$, and $q$ is a power of $2$ or $q\equiv \pm 3 \pmod 8$. 
\end{proposition}

\begin{definition}[{\cite[Definition 2.1]{OS}}]
 For any family $\F$ of subgroups of $G$ define 
 $$i_\F(H)=\frac{1}{[N_G(H):H]}(1-\chi(\K(\F_{>H}))).$$
\end{definition}

\begin{lemma}[{\cite[Lemma 2.3]{OS}}]\label{lemma2.3}
 Fix a separating family $\F$, a finite $H$-universal $(G,\F)$-complex $X$, and a subgroup $H\subseteq G$. For each $n$, let $c_n(H)$ denote the number of orbits of $n$-cells of type $G/H$ in $X$. Then $i_\F(H)=\sum_{n\geq 0} (-1)^nc_n(H)$.
\end{lemma}

\begin{proposition}[{\cite[Tables 2,3,4]{OS}}]\label{indices}
 Let $G$ be one of the simple groups $\PSL_2(2^k)$ for $k\geq 2$, $\PSL_2(q)$ for $q\equiv \pm 3 \pmod 8$ and $q\geq 5$, or $\Sz(2^k)$ for odd $k\geq 3$. Then $i_\SLV( 1 ) = 1$.
\end{proposition}



\section[The case of p-rank 3]{The case of $p$-rank 3}

\todo{Recordar que la acci�n de $G$ en $S_p(G)$ es por conjugaci�n.
%Equipped with the conjugation action of $G$, $X$ is a $G$-complex.
}
%Now using the results of Oliver and Segev \cite{OS} we prove Quillen's conjecture for groups of $p$-rank $3$.


\begin{theorem}\label{genthm}
If $X$ is an acyclic and $2$-dimensional $G$-invariant subcomplex of $\K(\S_p(G))$, then $O_p(G)\neq 1$.
%In particular, $\K(\S_p(G))$ is contractible.

\begin{proof}
Suppose $O_p(G)=1$. Then $G$ acts fixed point freely on $X$.
Consider the subgroup $N$ generated by the subgroups $N'\triangleleft G$ such that $X^{N'}\neq \emptyset$.
Clearly $N$ is normal in $G$.
By \cref{teoB} $Y=X^N$ is acyclic (in particular it is nonempty) and the action of $G/N$ on $Y$ is essential and fixed point free.
By \cref{lemma1.2} $\F=\{ H\leq G/N \tq Y^H\neq \emptyset\}$ is a separating family and $Y$ is an $H$-universal $(G/N,\F)$-complex.
Thus, \cref{teoA} asserts that $G/N$ must be one of the groups $\PSL_2(2^k)$ for $k\geq 2$, $\PSL_2(q)$ for $q\equiv \pm 3 \pmod 8$ and $q\geq 5$, or $\Sz(2^k)$ for odd $k\geq 3$.
In any case, by \cref{proposition6.4} we must have $\F=\SLV$.
By \cref{indices}, $i_\SLV(1)=1$.
Finally by \cref{lemma2.3}, $Y$ must have at least one free $G/N$-orbit.
Therefore $X$ has a $G$-orbit of type $G/N$. Let $\sigma=(A_0<\ldots <A_j)$ be a simplex of $X$ with stabilizer $N$.
Since $A_0\triangleleft N$, we have that $O_p(N)$ is nontrivial.
Since $N\triangleleft G$ and $O_p(N) \, \mathrm{char}\, N$ we have $O_p(N)\triangleleft G$ and therefore $O_p(N)\leq O_p(G)$.
So $O_p(G)$ is nontrivial, a contradiction.
\end{proof}

\end{theorem}


Since the $p$-rank of $G$ is equal to $\dim \K(\A_p(G))+1$ we obtain:

\begin{corollary}\label{prank3}
Let $G$ be a finite group of $p$-rank $3$. If $\wt{H}_*(\A_p(G))=0$ then $O_p(G)\neq 1$.
\end{corollary}


By \cref{genthm}, Quillen's conjecture also holds when $\K(\B_p(G))$ is $2$-dimensional. Recall that the subposet $\B_p(G) = \{Q\in \S_p(G) : Q = O_p(N_G(Q))\}$ is homotopy equivalent to $\S_p(G)$. See \cite{S} for an account of the relations between the different $p$-group complexes.

Finally we mention that a possible approach to prove \cref{QCAschbacherSmith} is to find an acyclic and $G$-invariant $2$-dimensional subcomplex of $\K(\S_p(G))$. If Quillen's conjecture were true, then this would be possible. Therefore, by \cref{genthm} we have the following equivalent version of the conjecture.

\begin{conjecture}[Restatement of Quillen's conjecture]
Assume $\K(\S_p(G))$ is acyclic. Then there exists a $G$-invariant acyclic subcomplex of $\K(\S_p(G))$ of dimension at most $2$.\todo{I: no me convence poner este restatement.}
\end{conjecture}

\section{Some examples}

\todo[inline]{I: ac� irian los ejemplos.}

$p=3$, $$G=(\Sz(8)^3 \times U_3(8) )\rtimes (C_3\times C_3)$$

...

\begin{thebibliography}{9}
\bibitem{AK} M. Aschbacher, P.B. Kleidman. \textit{On a conjecture of Quillen and a lemma of Robinson}. Arch. Math. 55 (1990), 209--217.
\bibitem{AschbacherSmith} M. Aschbacher, S.D. Smith. \textit{On Quillen's conjecture for the $p$-groups complex}. Ann. Math. (2) 137 (1993), 473--529.
\bibitem{AschbacherSegev} M. Aschbacher, Y. Segev. \textit{A fixed point theorem for groups acting on finite $2$-dimensional acyclic simplicial complexes}. Proc. London Math. Soc. (3) 67 (1993), 329--354.
\bibitem{Brown} K.S. Brown. \textit{Euler characteristics of groups: the $p$-fractional part}. Invent. Math. 29 (1975), 1--5.
\bibitem{CD} C. Casacuberta, W. Dicks. \textit{On finite groups acting on acyclic complexes of dimension two}. Publ. Mat. 36 (1992), 463--466.
\bibitem{OS} B. Oliver, Y. Segev. \textit{Fixed point free actions on $\Z$-acyclic $2$-complexes}. Acta. Math. 189 (2002), 203--285.
\bibitem{Q} D. Quillen. \textit{Homotopy properties of the poset of nontrivial p-subgroups of a group}. Advances in Mathematics 28 (1978), 101--128.
\bibitem{S} S.D. Smith. \textit{Subgroup complexes}. Mathematical Surveys and Monographs, 179. Amer. Math. Soc., Providence, RI. 2011. xii+364.
\end{thebibliography}
\end{document}