\documentclass[11pt,twoside]{amsart}
 \usepackage[T1]{fontenc}
\usepackage[utf8]{inputenc}
\usepackage[usenames]{color}
\usepackage{graphicx}
\usepackage{float}
\usepackage{fancyhdr}
\usepackage{geometry}
\usepackage{subcaption} 
 %\usepackage{amsrefs}
 %%%%%%%%%%%%%%%%%%%%%%%%%%%%%%%%Mathematical packages and others%%%%%%%%%%%%%%%%%%%%%%%%
\input{xy}
\xyoption{all}
\xyoption{poly}
\usepackage[all]{xy}

\setlength{\textwidth}{15cm}
\setlength{\topmargin}{0cm}
\setlength{\oddsidemargin}{.5cm}
\setlength{\evensidemargin}{.5cm}
\setlength{\textheight}{21.5cm}


 \usepackage{amsmath,amsthm,amsfonts,amssymb}
      \theoremstyle{plain}
      \newtheorem{theorem}{Theorem}[section]
      \newtheorem*{theorem*}{Theorem}
      \newtheorem{lemma}[theorem]{Lemma}
      \newtheorem{corollary}[theorem]{Corollary}
      \newtheorem{proposition}[theorem]{Proposition}
      \newtheorem{conjecture}[theorem]{Conjecture}
      \newtheorem*{aproposition}{Proposition \ref{propWebbConjReformulated}}
      \newtheorem*{bproposition}{Proposition \ref{propMxlRank}}
      \newtheorem*{atheorem}{Theorem \ref{theoremApGGcontractil}}
      \newtheorem*{aconjecture}{Conjecture \ref{strongerConjecture}}
      \theoremstyle{definition}
	  \newtheorem{example}[theorem]{Example}
      \newtheorem{definition}[theorem]{Definition}
      
      \theoremstyle{remark}
      \newtheorem{remark}[theorem]{Remark}

 \newcommand\CC{{\mathbb{C}}}
 \newcommand\RR{{\mathbb{R}}}
 \newcommand\QQ{{\mathbb{Q}}}
 \newcommand\ZZ{{\mathbb{Z}}}
 \newcommand\NN{{\mathbb{N}}}
 \newcommand\PP{{\mathbb{P}}}
 \newcommand\FF{{\mathbb{F}}}
 \renewcommand\SS{{\mathbb{S}}}
 \renewcommand\AA{{\mathbb{A}}}

 \def\colim{\operatorname{colim}}
 %\def\hocolim{{\mathrm {hocolim \hspace{2 pt}}}}
 \def\hocolim{\operatorname{\underline{hocolim}} }
 \def\thocolim{\operatorname{hocolim}} 
 \def\Top{{\mathrm {Top}}}
 \def \FP{{\mathcal P_{< \infty}}}
 
 \def\A{{\mathcal A}}
 \def\B{{\mathcal B}}
 \def\C{{\mathcal C}}
 \def\D{{\mathcal D}}
 \def\E{{\mathcal E}}
 \def\F{{\mathcal F}}
 \def\G{{\mathcal G}}
 \def\H{{\mathcal H}}
 \def\I{{\mathcal I}}
 \def\J{{\mathcal J}}
 \def\K{{\mathcal K}}
 \def\L{{\mathcal L}}
 \def\M{{\mathcal M}}
 \def\N{{\mathcal N}}
 \def\O{{\mathcal O}}
 \def\P{{\mathcal P}}
 \def\Q{{\mathcal Q}}
 \def\R{{\mathcal R}}
 \def\S{{\mathcal S}}
 \def\T{{\mathcal T}}
 \def\U{{\mathcal U}}
 \def\V{{\mathcal V}}
 \def\W{{\mathcal W}}
 \def\X{{\mathcal X}}
 \def\Y{{\mathcal Y}}
 \def\Z{{\mathcal Z}}
 
 \def\i{{\mathfrak{i}}}
 \def\s{{\mathfrak{s}}}
 
 \def\Id{{\text{Id}}}
 \def\id{{\text{id}}}
 \def\fix{{\text{Fix}}}
 \def\GL{{\text{GL}}}
 \def\PSL{{\text{PSL}}}
 \def\SL{{\text{SL}}}
 \def\Syl{{\text{Syl}}}
 \def\cte{{\text{const}}}
 \def\Max{{\text{Max}}}
 \def\join{*}
\def\Aut{{\text{Aut}}}
\def\Out{{\text{Out}}}

\newcommand{\weak}{\underset{w}{\approx}}
\newcommand{\normal}{\trianglelefteq}

% Espacio generador por []
\newcommand\gen[1]{\left\langle#1\right\rangle}

\usepackage{tikz}

\newcommand\threedef{\hspace{2 pt}\diagup\hspace{-4.8 pt} \searrow\hspace{-8pt}^3 \hspace{5 pt}}
\newcommand\gc{{\hspace{2.4 pt} \searrow\hspace{-8 pt}^{\gamma} \hspace{5 pt}}}
\newcommand\ce{{\hstomaremospace{2.4 pt} \searrow\hspace{-8 pt}^e \hspace{5 pt}}}
\newcommand\ee{{\hspace{3 pt} \nearrow\hspace{-13 pt}^e \hspace{8 pt}}}
\newcommand\se{{\hspace{2 pt}\diagup\hspace{-4.8 pt} \searrow\hspace{5 pt}}Pulkus J and Welker V., On the homotopy type of the p-subgroup complex
for finite solvable groups, J. Austral. Math. Soc, Series A. 69(2000), 212-
228.}
\newcommand\co{{\hspace{2 pt}\searrow \hspace{3 pt}}}
\newcommand\ex{{\nearrow \hspace{3 pt}}}
\newcommand\esc{{\searrow\hspace{-6 pt}\searrow\hspace{-8 pt}^e \hspace{5 pt}}}
\newcommand\sco{{\searrow \hspace{-6 pt}\searrow\hspace{3 pt}}}
\newcommand\stre{\sim \hspace {-9 pt}_{_{_{S e}}}}
\newcommand\we{\simeq\hspace {-11 pt}_{_{_{we}}}}
\newcommand\he{\simeq\hspace {-11 pt}_{_{_{he}}} }

      \makeatletter
      \def\@setcopyright{}
      \def\serieslogo@{}
      \makeatother

\begin{document}


Fix a finite group $G$ and a prime number $p$ dividing its order $|G|$. Denote by $\A_p(G)$ the poset of nontrivial elementary abelian $p$-subgroups of $G$ and by $\S_p(G)$ the poset of all nontrivial $p$-subgroups of $G$. Let $\Omega_1(G)$ denotes the subgroup generated by all the elements of order $p$ in $G$. We prove the following theorem by using the results of Aschbacher and Smith \cite{AS93}.

\begin{theorem}
If $G$ has $p$-rank at most $3$ and $O_p(G) = 1$, then $\tilde{H}_*(\A_p(G))\neq 0$.
\end{theorem}

We perform a serie of reductions first by using the results of Aschbacher and Smith. Recall that a component of $G$ is a subnormal quasisimple subgroup of $G$. A group $H$ is termed quasisimple if it is perfect and $H/Z(H)$ is simple. Equivalently, $H$ is a central extension of a simple group. The subgroup $E(G)$ of $G$ is the subgroup generated by all the components of $G$. It is known that different components of $G$ commute, and so $E(G)$ is the central product of the components of $G$. Denote by $\C(G)$ the set of components of $G$. Let $F(G)$ denotes the Fitting subgroup of $G$ and $F^*(G) = F(G)E(G)$ the generalized Fitting subgroup of $G$. Recall that $C_G(F^*(G))\leq F ^*(G)$. Denote by $r_p(G)$ the $p$-rank of the group $G$.

\begin{enumerate}
\item By \cite{Qui78} we may assume that $r_p(G) = 3$.
\item $\Omega_1(G) = G$ : since $\A_p(G) = \A_p(\Omega_1(G))$.
\item\label{opprime} $O_{p'}(G) = 1$: by \cite[Proposition 1.6]{AS93}.
\item By \ref{opprime} and the hypothesis $O_p(G) = 1$, $F(G) = 1$, so $F^*(G) = E(G) = L_1\times\ldots\times L_n$ is the direct product of the components of $G$, which are simple. See also \cite[Proposition 1.5]{AS93}.
\item $G$ does not have a strongly $p$-embedded subgroup. That is, $\A_p(G)$ is a connected poset (see \cite{Qui78}).
\item By \cite{AK90} we may assume that $G$ is not almost simple, so $n \geq 2$.
\item Note that $p\mid |L_i|$ for all $i$ since $O_{p'}(G) = 1$. In particular, $F^*(G) = L_1\times\ldots\times L_n$ contains an elementary abelian $p$-subgroup of rank $n$. Therefore, $n\leq 3$.
\item Since $n = 2$ or $3$, some $L_i$ has $p$-rank equal $1$. The only simple group of $2$-rank $1$ is $C_2$ (see \cite{Gor83}). Since $\gen{L_i : L_i\simeq C_2} \leq O_2(G) = 1$, we deduce $p \neq 2$.
\item By the above reasoning, $p$ is an odd prime, so the Sylow $p$-subgroups of those components of $p$-rank $1$, are cyclic.
\item If $G$ can be decomposed as a direct product $G_1 \times G_2$ with $p\mid |G_i|$ for $i = 1,2$, then $\A_p(G) \simeq \A_p(G_1) * \A_p(G_2)$ (see \cite{Qui78}), and we can apply inductive hypothesis. Therefore, we may assume $G$ is an indecomposable group.
\item $\Omega_1(F^*(G)) = F^*(G) < G = \Omega_1(G) $, so we can always get an element $x$ of order $p$ such that $x\in G - F ^*(G)$.
\end{enumerate}

At this point we divide the proof in two cases: $n = 2$ and $n = 3$.

\vspace{0.3cm}

\textit{Assume $F^*(G) = L_1\times L_2\times L_3$.}

Therefore, $r_p(L_i)  = 1$ for all $i$. If $x\in G - F^*(G)$ is an element of order $p$, then $x$ acts in the set $\C(G)$ of components of $G$. Since $x$ has primer order $p$, the action is either regular and $p = 3$ or $x$ normalizes each component $L_i$.

If $x$ normalizes each component $L_i$, then $x$ acts on the set of Sylow $p$-subgroups $\Syl_p(L_i)$ for each $i$. Since the number of Sylow $p$-subgroups is coprime with $p$, there exists a Sylow $S_i\in\Syl_p(L_i)$ normalized by $x$, for each $i = 1,2,3$. Therefore, there is an element $y_i\in S_i$ of order $p$, such that $[x,y_i] = 1$. Hence, $\gen{x,y_1,y_2,y_3}$ is an elementary abelian $p$-subgroup of $G$ of rank $4$, a contradiction.

Thus, none $x \in G - F^*(G)$ of order $p$ can normalize the components of $G$. In particular, $p = 3$ and the action of $x$ in $F^*(G)$ is regular on the components. That is, $F^*(G) \simeq L_1 \wr \gen{x}$. We may assume $L_i ^x = L_{i+1}$ for all $i$, with $i + 1  =1$ if $i = 3$.

Recall the following result of \cite{Asc86}.

\begin{proposition}\cite[31.18.1]{Asc86}.
Let $O_{p'}(G) = 1$, $x$ of order $p$ in $G$, $L\in \C(G)$, and $Y = O_{p'}(C_G(x))$. If $L \neq [L,x]$ then $[L,Y] = 1$ and one of the following holds:
\begin{enumerate}
\item $L\in \C(C_G(x))$, or
\item $L\neq L^x$ and $C_{[L,x]}(x)' = K\in \C(C_G(x))$ with $K$ a homomorphic image of $L$.
\end{enumerate}
\end{proposition}

We apply this result to our situation with $L= L_1$. If $l\in L_1$, then $[l,x] = lxl^{-1}x^{-1}$. Note that $u = xl^{-1} x^{-1}\in L_ 2$ since ${}^{x}L_1 = L_1^{x^2} = L_2$. Thus, $[l,x] = l u $. If $[L,x]\leq L_1$ then $u\in L_1\cap L_2  = 1$.  That is, $u = {}^xl = 1$, which means $l = 1$, a contradiction since $l\in L_1$ is any element. Therefore $L_1\neq [L_1,x]$ and we are in the hypotheses of the above proposition. Since $L_1\not\leq C_G(x)$ because of the action of $x$, the second case of the proposition holds and $C_{[L_1,x]}(x) ' = K\in \C(C_G(x))$ and $K$ is a homomorphic image of $L_1$. Given that $L_1$ is a simple group and that the components are nontrivial groups, we deduce $K =L_1$ and $L_1\leq C_{[L_1,x]}(x) ' \leq C_G(x)$, a contradiction. 

\vspace{0.3cm}


\textit{Assume $F^*(G) = L_1\times L_2$}

In this case, $p\geq 3$, so any order $p$ element $x\in G - F^*(G)$ normalizes $L_1$ and $L_2$ the components of $G$. Thus, $L_i\normal G$ for $i = 1,2$.  Note that if $r_p(L_i) \geq 2$, then we may construct an elementary abelian $p$-subgroup of $p$-rank $4$ just as the previous case. Therefore, $r_p(L_i) = 1$ and each $L_i$ has a strongly $p$-embedded subgroup.

Assume that $H_1(\A_p(G)) = 0$.  By the proof of \cite[10.3]{Asc93}, $G = (L_1\times L_2)X$ for some subgroup $X\leq G$ of order $p$ inducing outer automorphisms on $L_1$ and $L_2$. Moreover, $L_i$ is of Lie type and Lie rank $1$ in characteristic $p$ and $X$ induces field automorphisms. That is, $L_i \simeq L_2(q)$, $U_3(q^p)$ or ${}^2G_2(q)$ with $q$ a power of $p$. Since each $L_i$ has $p$-rank $1$, 
their Sylow $p$-subgroups must by cyclic. Therefore, $L_i \simeq L_2(p)$ and $p>3$ or $L_i\simeq {}^2G_2(3)'$ and $p = 3$. 
Since $\Out(L_2(p)) = C_2$, $L_2(p)$ does not have outer automorphisms of primer order $p > 3$. Therefore $L_i\simeq {}^2G_2(3)'$ and $p = 3$. Note that ${}^2G_2(3)' \simeq \PSL_2(8)$. However, in this case $\A_p(F^*(G)) \simeq \A_p(G)$ are homotopy equivalent, and $H_1(\A_p(F^*(G))) \neq 0$, contradicting our assumption (see the proof of \cite[10.3]{Asc93}).



\begin{thebibliography}{}

\bibitem[AK90]{AK90} M. Aschbacher, P. B. Kleidman. On a conjecture of Quillen and a lemma of Robinson. Arch. Math. (Basel) 55 (1990), no. 3, 209-217.

\bibitem[AS93]{AS93} M. Aschbacher, S. D. Smith. On Quillen's conjecture for the $p$-groups complex. Ann. of Math. (2) 137 (1993), no. 3, 473-529.

\bibitem[Asc86]{Asc86} M. Aschbacher. Finite group theory, Cambridge University Press, Cambridge, 2000, xii+304.

\bibitem[Asc93]{Asc93} M. Aschbacher. Simple connectivity of $p$-group complexes. Israel J. Math. 82 (1993), no. 1-3, 1-43.

\bibitem[Gor83]{Gor83} D. Gorenstein. The Classification of Finite Simple Groups, Volume 1: Groups of Noncharacteristics 2 Type-Springer US (1983).

\bibitem[Qui78]{Qui78} D. Quillen. Homotopy properties of the poset of nontrivial $p$-subgroups of a group. Adv. Math. 28 (1978), 101–128.
\end{thebibliography}

\end{document}


